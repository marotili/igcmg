%
% Version 0.1 der Statuten der "GADGETs"
%
% 20/02/2015
%

\documentclass[a4paper,12pt]{article}
\usepackage{times}
\usepackage{t1enc}
\usepackage[austrian]{babel}
\usepackage[T1]{fontenc}
\usepackage{enumerate}
\usepackage{vmargin}
\setpapersize{A4}
\usepackage[utf8]{inputenc}

\setmarginsrb{1.75cm}{1.75cm}{1.75cm}{1.75cm}%
             {0pt}{0pt}
             {\baselineskip}{2.0\baselineskip}
\parindent=1em
\parskip=0pt
\setlength{\marginparwidth}{4.0cm}
\setlength{\marginparsep}{1.25em}

\hyphenation{Ver-einsG be-stim-mten dis-zi-pli-nä-ren}

\makeatletter
\renewcommand{\@seccntformat}[1]{§\,\csname the#1\endcsname:\space}
\renewcommand{\labelenumi}{(\theenumi)}
\renewcommand{\p@enumii}{}
\makeatother

\newcommand{\comment}[1]{{\bf /*Kommentar:} #1 {\bf Ende Kommentar*/}}
% comment the following line to get comments:
% \def\comment#1{}

\newenvironment{Itemize}{%
  \begin{itemize}\setlength{\parsep}{\parskip}\setlength{\itemsep}{0pt}%
}{\end{itemize}}

\begin{document}
\begin{center}
  \Large%
   Statuten des Vereins\\[.1\baselineskip]
  \textsc{\LARGE Game Developers Graz - Verein zur Förderung und Vernetzung lokaler SpieleentwicklerInnen}

%
% Alternativ
%	Verein zur Förderung von Spieleentwicklung und Spielkultur
%
%
%

  \vskip\baselineskip%
\end{center}

\section{Name, Sitz und Tätigkeitsbereich} % § 1
\label{sec:Name}
\begin{enumerate}
\item Der Verein führt den Namen "Game Developers Graz - Verein zur F\"orderung und Vernutzung lokaler SpieleentwicklerInnen" kurz: GADGet.

\item Er hat seinen Sitz in Graz und erstreckt seine Tätigkeit auf das Gebiet des Bundeslandes Steiermark.
\item Die Errichtung von Zweigvereinen ist nicht beabsichtigt.
\end{enumerate}

\section{Zweck} % § 2
\label{sec:Zweck}

\section{Mittel zur Erreichung des Vereinszwecks}% § 3
\label{sec:Mittel}
\begin{enumerate}
\item Der Vereinszweck soll durch die in den Abs. \ref{item:zweck-ideelle-mittel} und \ref{item:zweck-materielle-mittel} angeführten ideellen und materiellen Mittel erreicht werden.
\item\label{item:zweck-ideelle-mittel} Als ideelle Mittel dienen:
\begin{enumerate}[(a)]
\item Regelmäßige Treffen
\item Vortragsveranstaltungen und Versammlungen, Diskussionsabende
\item Betrieb einer Webseite und "offentliche Auftritte
\end{enumerate}
\item\label{item:zweck-materielle-mittel} Die erforderlichen materiellen Mittel sollen aufgebracht werden durch:
\begin{enumerate}[(a)]
\item Sponsoringvereinbarungen (Werbung im Rahmen von Vereinsveranstaltungen)
\item Erträgen aus Veranstaltungen, vereinseigenen Unternehmungen
\item Spenden, Sammlungen, Vermächtnisse und sonstige Zuwendungen
\item Förderungen und Subventionen \comment{Wird das noch irgendwo erw"ahnt?}
\end{enumerate}
\end{enumerate}


\section{Arten der Mitgliedschaft} % § 4
\label{sec:Mitgliedschaft-Arten}
\begin{enumerate}
\item Die Mitglieder des Vereins gliedern sich in ordentliche, außerordentliche und Ehrenmitglieder.
\item Ordentliche Mitglieder sind jene, die sich voll an der Vereinsarbeit beteiligen. Außerordentliche Mitglieder sind solche, die unterstützend in der Vereinsarbeit mitwirken. Ehrenmitglieder sind Personen, die hiezu wegen besonderer Verdienste um den Verein ernannt werden.
\end{enumerate}

\section{Erwerb der Mitgliedschaft} % § 5
\begin{enumerate}
\item Mitglieder des Vereins können alle physischen und juristische Personen, sowie rechtsfähige Personengesellschaften werden
\item Über die Aufnahme von ordentlichen Mitgliedern entscheiden der Vorstand und die ordentlichen Mitglieder. Die Aufnahme kann ohne Angabe von Gründen verweigert werden. Über die Aufnahme
von außerordentlichen Mitgklieder entscheidet der Vorstand. Als ordentliches Mitglied kann
nur aufgenommem werden, wer Unterst"utzung von 50\% der ordentlichen Mitglieder hat.
\item Die Ernennung zum Ehrenmitglied erfolgt auf Antrag des Vorstandes durch die Generalversammlung.
\item Vor Konstituierung des Vereins erfolgt die vorläufige Aufnahme von Mitgliedern durch den / die Proponenten. Diese Mitgliedschaft wird erst mit Konstituierung des Vereins wirksam.
\end{enumerate}

\section{Beendigung der Mitgliedschaft} % § 6
\begin{enumerate}
\item Die Mitgliedschaft erlischt durch Tod, bei juristischen Personen und rechtsfähigen Personengesellschaften durch Verlust der Rechtspersönlichkeit, durch freiwilligen Austritt und durch Ausschluss.
\item Der Austritt kann nur am Ersten eines Monats erfolgen. Er muss dem Vorstand mindestens 1 Monat vorher schriftlich mitgeteilt werden. Erfolgt die Anzeige verspätet, so ist sie erst zum nächsten Austrittstermin wirksam. Für die Rechtzeitigkeit ist das Datum der Postaufgabe maßgeblich.
\item Der Vorstand kann ein Mitglied ausschließen, wenn dieses trotz zweimaliger schriftlicher Mahnung unter Setzung einer angemessenen Nachfrist länger als sechs Monate mit der Zahlung der Mitgliedsbeiträge im Rückstand ist. Die Verpflichtung zur Zahlung der fällig gewordenen Mitgliedsbeiträge bleibt hiervon unberührt. 
\item Der Ausschluss eines Mitglieds aus dem Verein kann vom Vorstand auch wegen grober Verletzung anderer Mitgliedspflichten und wegen unehrenhaften Verhaltens verfügt werden.
\item Die Aberkennung der Ehrenmitgliedschaft kann aus den im Abs. 4 genannten Gründen von der Generalversammlung über Antrag des Vorstands beschlossen werden
\end{enumerate}

\section{Rechte und Pflichten der Mitglieder} % § 7
\label{sec:rechte-und-pflichten}
\begin{enumerate}
\item\label{item:mitglieder-stimmrecht} Die Mitglieder sind berechtigt, an allen Veranstaltungen des Vereines teilzunehmen und die Einrichtungen des Vereins zu beanspruchen. Das Stimmrecht in der Generalversammlung sowie das aktive und passive Wahlrecht steht nur den ordentlichen Mitgliedern zu.
\item Die Mitglieder sind verpflichtet, die Interessen des Vereins nach Kräften zu fördern und alles zu unterlassen, wodurch das Ansehen und der Zweck des Vereins Abbruch erleiden könnte. Sie haben die Vereinsstatuten und die Beschlüsse der Vereinsorgane zu beachten.
\end{enumerate}

\section{Vereinsorgane} % § 8
\begin{enumerate}
\item Organe des Vereins sind die Generalversammlung (§§ 9 und 10), der Vorstand (§§ 11 bis 13), die Rechnungsprüfer (§ 14) und das Schiedsgericht (§ 15).
\end{enumerate}

\section{Generalversammlung} % § 9
\label{sec:generalversammlung}
\begin{enumerate}
\item Die ordentliche Generalversammlung findet jährlich statt.
\item Eine außerordentliche Generalversammlung findet auf Beschluss des Vorstandes, der ordentlichen Generalversammlung oder auf schriftlichen begründeten Antrag von mindestens einem F"unftel der stimmberechtigten (§ \ref{sec:rechte-und-pflichten} Abs. \ref{item:mitglieder-stimmrecht} und § \ref{sec:generalversammlung} Abs. \ref{item:gv-stimmrecht}) Mitglieder oder auf Verlangen der Rechnungsprüfer binnen vier Wochen statt.
\item Sowohl zu den ordentlichen wie auch zu den außerordentlichen Generalversammlungen sind alle Mitglieder mindestens zwei Wochen vor dem Termin schriftlich einzuladen. Die Anberaumung der Generalversammlung hat unter Angabe der Tagesordnung zu erfolgen. Die Einberufung erfolgt durch den Vorstand.
\item Anträge zur Generalversammlung sind mindestens drei Tage vor dem Termin der Generalversammlung beim Vorstand schriftlich einzureichen.
\item Gültige Beschlüsse - ausgenommen solche über einen Antrag auf Einberufung einer außerordentlichen Generalversammlung - können nur zur Tagesordnung gefasst werden.
\item \label{item:gv-stimmrecht} Bei der Generalversammlung sind alle Mitglieder teilnahmeberechtigt. Stimmberechtigt sind nur die ordentlichen Mitglieder. Jedes Mitglied hat eine Stimme. Juristische Personen werden durch einen Bevollmächtigten vertreten. Die Übertragung des Stimmrechtes auf ein anderes Mitglied im Wege einer schriftlichen oder elektronisch unterzeichneten fernschriftlichen Bevollmächtigung ist zulässig.
\item Die Generalversammlung ist bei Anwesenheit der Hälfte aller stimmberechtigten Mitglieder bzw. ihrer Vertreter (Abs. 6) beschlussfähig. Ist die Generalversammlung zur festgesetzten Stunde nicht beschlussfähig, so findet die Generalversammlung 30 Minuten später mit derselben Tagesordnung statt, die beschlussfähig ist, wenn mindestens drei stimmberechtigte Mitglieder anwesend sind.
\item Die Wahlen und die Beschlussfassungen in der Generalversammlung erfolgen in der Regel mit einfacher Stimmenmehrheit. Beschlüsse, mit denen das Statut des Vereins geändert oder der Verein aufgelöst werden soll, bedürfen jedoch einer qualifizierten Mehrheit von zwei Dritteln der abgegebenen gültigen Stimmen.
\item Den Vorsitz in der Generalversammlung führt der Obmann, in dessen Verhinderung sein Stellvertreter. Wenn auch dieser verhindert ist, so führt das an Jahren längste Vorstandsmitglied im Verein, bei Gleichheit das in Jahren älteste anwesende Vorstandsmitglied den Vorsitz.
\end{enumerate}

\section{Aufgaben der Generalversammlung} % § 10
\begin{enumerate}
\item Der Generalversammlung sind folgende Aufgaben vorbehalten:
	\begin{enumerate}[(a)]
	\item Entgegennahme und Genehmigung des Rechenschaftsberichtes und des Rechnungsabschlusses;
	\item Beschlussfassung über den Voranschlag;
	\item Wahl, Bestellung und Enthebung der Mitglieder des Vorstandes und der Rechnungsprüfer; Genehmigung von Rechtsgeschäften zwischen Vorstandsmitgliedern und Rechnungsprüfern mit dem Verein;
	\item Entlastung des Vorstandes;
	\item Verleihung und Aberkennung der Ehrenmitgliedschaft;
	\item Beschlussfassung über Statutenänderungen und die freiwillige Auflösung des Vereines;
	\item Beratung und Beschlussfassung über sonstige auf der Tagesordnung stehende Fragen.
	\end{enumerate}
\end{enumerate}

\section{Vorstand} % § 11
\begin{enumerate}
\item Der Vorstand besteht aus sechs Mitgliedern, und zwar aus Obmann und Stellvertreter, Schriftführer und Stellvertreter sowie Kassier und Stellvertreter.
\item Der Vorstand wird von der Generalversammlung gewählt. Der Vorstand hat bei Ausscheiden eines gewählten Mitglieds das Recht, an seine Stelle ein anderes wählbares Mitglied zu kooptieren, wozu die nachträgliche Genehmigung in der nächstfolgenden Generalversammlung einzuholen ist. Fällt der Vorstand ohne Selbstergänzung durch Kooptierung überhaupt oder auf unvorhersehbar lange Zeit aus, so ist jeder Rechnungsprüfer verpflichtet, unverzüglich eine außerordentliche Generalversammlung zum Zweck der Neuwahl eines Vorstands einzuberufen. Sollten auch die Rechnungsprüfer handlungsunfähig sein, hat jedes ordentliche Mitglied, das die Notsituation erkennt, unverzüglich die Bestellung eines Kurators beim zuständigen Gericht zu beantragen, der umgehend eine außerordentliche Generalversammlung einzuberufen hat.
\item \label{item:funktionsperiode} Die Funktionsperiode des Vorstands beträgt ein Jahr; Wiederwahl ist möglich.
\item Der Vorstand wird vom Obmann/von der Obfrau, bei Verhinderung von seinem/seiner/ihrem/ihrer Stellvertreter/in, schriftlich oder mündlich einberufen. Ist auch diese/r auf unvorhersehbar lange Zeit verhindert, darf jedes sonstige Vorstandsmitglied den Vorstand einberufen.
\item Der Vorstand ist beschlussfähig, wenn alle seine Mitglieder eingeladen wurden und mindestens die Hälfte von ihnen anwesend ist.
\item Der Vorstand fasst seine Beschlüsse mit einfacher Stimmenmehrheit; bei Stimmengleichheit gibt die Stimme des/der Vorsitzenden den Ausschlag.
\item Den Vorsitz führt der Obmann, bei Verhinderung sein Stellvertreter. Ist auch dieser verhindert, obliegt der Vorsitz dem an Jahren längsten Mitglied im Verein, bei Gleichheit dem ältesten anwesenden Vorstandsmitglied. \comment{Alter doch noch drinnen, weil wir alle gleich lang im Verein sind}
\item Außer durch den Tod und Ablauf der Funktionsperiode (Abs. \ref{item:funktionsperiode}) erlischt die Funktion eines Vorstandsmitglieds durch Enthebung (Abs. \ref{item:enthebung}) und Rücktritt (Abs. \ref{item:ruecktritt}).
\item \label{item:enthebung} Die Generalversammlung kann jederzeit den gesamten Vorstand oder einzelne seiner Mitglieder entheben. Die Enthebung tritt mit Bestellung des neuen Vorstands bzw. Vorstandsmitglieds in Kraft.
\item \label{item:ruecktritt} Die Vorstandsmitglieder können jederzeit schriftlich ihren Rücktritt erklären. Die Rücktrittserklärung ist an den Vorstand, im Falle des Rücktritts des gesamten Vorstands an die Generalversammlung zu richten. Der Rücktritt wird erst mit Wahl bzw. Kooptierung (Abs. 2) eines Nachfolgers wirksam.
\end{enumerate}

\section{Aufgaben des Vorstands} % § 12
\begin{enumerate}
\item Dem Vorstand obliegt die Leitung des Vereines. Ihm kommen alle Aufgaben zu, die nicht durch die Statuten einem anderen Vereinsorgan zugewiesen sind. In seinen Wirkungsbereich fallen insbesondere folgende Angelegenheiten:
	\begin{enumerate}[(a)]
	\item Erstellung des Jahresvoranschlages sowie Abfassung des Rechenschaftsberichtes und des Rechnungsabschlusses;
	\item Vorbereitung der Generalversammlung;
	\item Einberufung der ordentlichen und der außerordentlichen Generalversammlung;
	\item Verwaltung des Vereinsvermögens;
	\item Ausschluss von Vereinsmitgliedern;
	\item Aufnahme und Kündigung von Angestellten des Vereines.
	\end{enumerate}
\end{enumerate}

\section{Besondere Obliegenheiten einzelner Vorstandsmitglieder} % § 13
\begin{enumerate}
\item Der Obmann führt die laufenden Geschäfte des Vereins. Der Schriftführer unterstützt den Obmann bei der Führung der Vereinsgeschäfte.
\item \label{item:vertreten} Der Obmann vertritt den Verein nach außen. Schriftliche Ausfertigungen des Vereins bedürfen zu ihrer Gültigkeit der Unterschriften des Obmanns und des Schriftführers, in Geldangelegenheiten (=vermögenswerte Dispositionen) des Obmanns und des Kassiers. Rechtsgeschäfte zwischen Vorstandsmitgliedern und dem Verein bedürfen zu ihrer Gültigkeit außerdem der Genehmigung der Generalversammlung.
\item Rechtsgeschäftliche Bevollmächtigungen, den Verein nach außen zu vertreten bzw. für ihn zu zeichnen, können ausschließlich von den in Abs. \ref{item:vertreten} genannten Funktionären erteilt werden. Eine elektronisch unterzeichnete fernschriftliche Vollmacht ist zulässig.
\item Bei Gefahr im Verzug ist der Obmann berechtigt, auch in Angelegenheiten, die in den Wirkungsbereich der Generalversammlung oder des Vorstandes fallen, unter eigener Verantwortung selbständig Anordnungen zu treffen; diese bedürfen jedoch der nachträglichen Genehmigung durch das zuständige Vereinsorgan.
\item Der Obmann führt den Vorsitz in der Generalversammlung und im Vorstand.
\item Der Schriftführer hat den Obmann bei der Führung der Vereinsgeschäfte zu unterstützen. Ihm obliegt die Führung der Protokolle der Generalversammlung und des Vorstandes.g
\item Der Kassier ist für die ordnungsgemäße Geldgebarung des Vereines verantwortlich.
\item Im Falle der Verhinderung treten an die Stelle des Obmannes, des Schriftführers und des Kassiers ihre Stellvertreter.
\end{enumerate}

\section{Rechnungsprüfer} % § 14
\begin{enumerate}
\item Die zwei Rechnungsprüfer werden von der Generalversammlung auf die Dauer von zwei Jahren gewählt. Wiederwahl ist möglich. Die
Rechnungsprüferinnen/Rechnungsprüfer dürfen keinem Organ - mit Ausnahme
der Generalversammlung - angehören, dessen Tätigkeit Gegenstand der
Prüfung ist. \comment{Update}
\item Den Rechnungsprüfern obliegt die laufende Geschäftskontrolle und die Überprüfung des Rechnungsabschlusses. Sie haben der Generalversammlung über das Ergebnis der Überprüfung zu berichten.
\item Im Übrigen gelten für die Rechnungsprüfer die Bestimmungen des § 11 Abs. 3, 8, 9 und 10 sowie des § 13 Abs. 2 letzter Satz sinngemäß.
\end{enumerate}

\section{Schiedsgericht} % § 15
\begin{enumerate}
\item Zur Schlichtung von allen aus dem Vereinsverhältnis entstehenden Streitigkeiten ist das vereinsinterne Schiedsgericht berufen. Es ist eine „Schlichtungseinrichtung“ im Sinne des Vereinsgesetzes 2002 und kein Schiedsgericht nach den §§ 577 ff ZPO.
\item Das Schiedsgericht setzt sich aus drei ordentlichen Vereinsmitgliedern zusammen. Es wird derart gebildet, dass ein Streitteil dem Vorstand ein Mitglied als Schiedsrichter schriftlich namhaft macht. Über Aufforderung durch den Vorstand binnen sieben Tagen macht der andere Streitteil innerhalb von 14 Tagen seinerseits ein Mitglied des Schiedsgerichts namhaft. Nach Verständigung durch den Vorstand innerhalb von sieben Tagen wählen die namhaft gemachten Schiedsrichter binnen weiterer 14 Tage ein drittes ordentliches Mitglied zum/zur Vorsitzenden des Schiedsgerichts. Bei Stimmengleichheit entscheidet unter den Vorgeschlagenen das Los. Die Mitglieder des Schiedsgerichts dürfen keinem Organ – mit Ausnahme der Generalversammlung – angehören, dessen Tätigkeit Gegenstand der Streitigkeit ist.
\item Das Schiedsgericht fällt seine Entscheidung nach Gewährung beiderseitigen Gehörs bei Anwesenheit aller seiner Mitglieder mit einfacher Stimmenmehrheit. Es entscheidet nach bestem Wissen und Gewissen. Seine Entscheidungen sind vereinsintern endgültig.
\end{enumerate}

\section{Freiwillige Auflösung des Vereins} % § 16
\begin{enumerate}
\item Die freiwillige Auflösung des Vereines kann nur in einer zu diesem Zweck einberufenen außerordentlichen Generalversammlung und nur mit Zweidrittelmehrheit der abgegebenen gültigen Stimmen beschlossen werden.
\item Diese Generalversammlung hat auch - sofern Vereinsvermögen vorhanden ist - über die Liquidation zu beschließen. Insbesondere hat sie einen Liquidator zu berufen und Beschuss darüber zu fassen, wem dieser das nach Abdeckung der Passiven verbleibende Vereinsvermögen zu übertragen hat.
\item Bei Auflösung des Vereines oder bei Wegfall des bisherigen begünstigten Vereinszweckes ist das verbleibende Vereinsvermögen für gemeinnützige oder mildtätige Zwecke im Sinne der §§ 34ff BAO zu verwenden.
\end{enumerate}

\end{document}
